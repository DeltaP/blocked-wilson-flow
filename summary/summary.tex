\documentclass[11pt]{article}
\usepackage[margin=1in]{geometry}
\usepackage{graphicx}
\usepackage{caption}
\usepackage{subcaption}
\usepackage{color}

\begin{document}

\title{Wilson Flow MCRG}
\author{Gregory Petropoulos}
\date{\today}
\maketitle

\section{Background}

The idea is to use wilson flow to move the configuration toward the renormalized trajectory before blocking.
The hope is that because we start blocking on the renormalized trajectory we can use the same blocking scheme for the entire analysis and get a unique step scaling function.

\section{Procedure}


\section{Partial Results for $\Delta\beta$}

As before some values of t match off of the range of $\beta$ that we currently have on the smaller volumes.  
The relevant matching is contained in the range of small $\beta$.

\section{Partial Results for optimal smearing time}

Figures \ref{fig:7.0_0}-\ref{fig:7.0_4} are the plots showing how matching is affected by smearing for matching on $\beta = 7.0$.
Figures \ref{fig:7.4_0} - \ref{fig:7.4_4} are the plots showing how matching is affected by smearing for $\beta=7.4$.
In all cases by the time I match the large volume blocked three times ({\color{blue}blue}) to the small volume blocked twice the smearing is no longer relevant to finding delta beta.
Also, unlike with regular MCRG here I have also shown the matching for the large volume blocked once ({\color{red}red}) and the small volume not blocked at all.

\begin{figure}[htpb]
  \centering
  \includegraphics[width=4in]{{../Plots/smearing_time/7.0_0}.png}
  \caption{$\beta = 7.0$ loop 0}
  \label{fig:7.0_0}
\end{figure}

\begin{figure}[htpb]
  \centering
  \includegraphics[width=4in]{{../Plots/smearing_time/7.0_1}.png}
  \caption{$\beta = 7.0$ loop 1}
  \label{fig:fig:7.0_1}
\end{figure}

\begin{figure}[htpb]
  \centering
  \includegraphics[width=4in]{{../Plots/smearing_time/7.0_2}.png}
  \caption{$\beta = 7.0$ loop 2}
  \label{fig:fig:7.0_2}
\end{figure}

\begin{figure}[htpb]
  \centering
  \includegraphics[width=4in]{{../Plots/smearing_time/7.0_3}.png}
  \caption{$\beta = 7.0$ loop 3}
  \label{fig:7.0_3}
\end{figure}

\begin{figure}[htpb]
  \centering
  \includegraphics[width=4in]{{../Plots/smearing_time/7.0_4}.png}
  \caption{$\beta = 7.0$ loop 4}
  \label{fig:7.0_4}
\end{figure}


\begin{figure}[htpb]
  \centering
  \includegraphics[width=4in]{{../Plots/smearing_time/7.4_0}.png}
  \caption{$\beta = 7.4$ loop 0}
  \label{fig:7.4_0}
\end{figure}

\begin{figure}[htpb]
  \centering
  \includegraphics[width=4in]{{../Plots/smearing_time/7.4_1}.png}
  \caption{$\beta = 7.4$ loop 1}
  \label{fig:7.4_1}
\end{figure}

\begin{figure}[htpb]
  \centering
  \includegraphics[width=4in]{{../Plots/smearing_time/7.4_2}.png}
  \caption{$\beta = 7.4$ loop 2}
  \label{fig:7.4_2}
\end{figure}

\begin{figure}[htpb]
  \centering
  \includegraphics[width=4in]{{../Plots/smearing_time/7.4_3}.png}
  \caption{$\beta = 7.4$ loop 3}
  \label{fig:7.4_3}
\end{figure}

\begin{figure}[htpb]
  \centering
  \includegraphics[width=4in]{{../Plots/smearing_time/7.4_4}.png}
  \caption{$\beta = 7.4$ loop 4}
  \label{fig:7.4_4}
\end{figure}

\section{In Progress and To Do}
\subsection{In Progress} 
\begin{enumerate}
  \item Generating $\beta=8.0$ for 24x48 m = 0.0
  \item Generating $\beta=7.6$ for 12x24 m = 0.0
\end{enumerate}

\subsection{To Do}

\begin{enumerate}
  \item Measure $\beta=8.0$ for 24x48 m = 0.0
  \item Measure $\beta=7.4$, 7.6 for 12x24 m = 0.0
  \item Add finding t optimal and $\Delta\beta$ optimal to script
  \item Add full jackknife analysis
\end{enumerate}
\end{document}
